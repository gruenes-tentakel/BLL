\newgeometry{margin=2.5cm}
\begin{titlepage}
\thispagestyle{empty}
\newcommand{\HRule}{\rule{\linewidth}{0.5mm}}
\hspace{1cm}
\center

\textsc{\huge Besondere Lernleistung}\\[2.0cm]
\textsc{\Large Paulsen-Gymnasium}\\[0.8cm]
\textsc{\Large Berlin-Steglitz}\\[0.8cm]
\MSonehalfspacing

\HRule\\%[1.4cm]
\MSdoublespacing
{ \huge \bfseries Die Verbindung des \\
pythagoreischen Lehrsatzes mit dem Großen Fermatschen Satz unter Beachtung der historischen Entwicklung}\\[0.2cm]
\HRule \\[2.4cm]
\MSonehalfspacing

\raggedright
\item[{Name:}] Charlotte Specht
\item[{Betreuende Lehrkraft:}] Herr Wessel
\item[{Datum:}] 06.12.2021
\item[{Hauptfach:}] Mathematik
\item[{Bezugsfach:}] Geschichte

\end{titlepage}
\restoregeometry

\setstretch{1.9}
\microtypesetup{protrusion=false}
\tableofcontents
\microtypesetup{protrusion=true}
\thispagestyle{empty}

\MSonehalfspacing
\newpage
\pagestyle{fancy}
\setcounter{page}{3}
